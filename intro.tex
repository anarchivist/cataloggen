\chapter*{Introduction}
\addcontentsline{toc}{chapter}{Introduction} 

\subsection*{The Weinberger ornithology collection}

An avid book collector and bird watcher, as of January 2009 Marc Weinberger had
built a book collection of more than eight thousand volumes, roughly one quarter
of which made up his ornithology collection. The present catalog represents a
description of the core of that ornithology collection, some 1900 volumes kept in
the library at the Weinberger residence in New Rochelle, NY.

The especial strength of the collection is in field guides, with secondary emphasis
on reference works and ornithological catalogs of the twentieth century.
The collection covers all parts of the globe, from Antarctica to the Antilles and
India to Indiana, with particular emphasis on American birds.

Of the 1317 works listed in this catalog, there is not one which does not deserve
at least some brief mention, but -- in the interests of space -- we must content
ourselves with listing just a few of the highlights:
\begin{itemize}
\item A first edition of Roger Tory Peterson's influential \textit{Field Guide to
	the Birds} (\ref{723})
\item No. 148 of 350 copies printed of the \textit{Sunset Edition De Luxe} (one
	of the less common editions of this beautiful work) of William Leon Dawson's
	\textit{Birds of California} (\ref{220})
\item Several items from the collection of the esteemed ornithologist Rodolphe
	Meyer de Schauensee (\ref{543}-\ref{544}, \ref{638}, \ref{670}, \ref{893}, \ref{943})
\item Limited edition of Herbert L. Stoddard's \textit{Bobwhite Quail}, including
	a pencil-signed etching by Frank W. Benson (\ref{511})
\item The earliest work in the collection, the 1831 edition of Alexander Wilson's
	\textit{American Ornithology} (\ref{23})
\end{itemize}

\subsection*{Preparation of the catalog}

The cataloging of this collection was completed by Jared Camins-Esakov and Catey
Farley with the help of Sarah Simms over a three month period between January
and March 2009. The cataloging was done in
{\ddag}biblios.net\footnote{\url{http://www.biblios.net/}}, a free service provided
by LibLime\footnote{LibLime, a provider of open source software for libraries,
\url{http://www.liblime.com/}} that enabled us to use catalog records from other
libraries as a starting point rather than needing to enter in all the common
data (e.g., title, author, series) ourselves.

Following the completion of the cataloging, we converted the records into MARCXML,
a standard administered by the Library of
Congress\footnote{\url{http://www.loc.gov/standards/marcxml/}}. The records were
then converted from MARCXML to \LaTeX\footnote{An open source system supported by
the \LaTeX project, \url{http://www.latex-project.org/}}, a document preparation
system for high-quality typesetting system, using XSLT\footnote{eXtensible
Stylesheet Language Transformations, \url{http://www.w3.org/TR/xslt20/}}, a
technology for transforming data in XML\footnote{See \emph{A Gentle Introduction
to XML} from the Text Encoding Initiative at
\url{http://www.tei-c.org/Guidelines/P4/html/SG.html} for more information about
XML.} formats such as MARCXML to other formats (in this case, \LaTeX). The document
was then typeset into a PDF\footnote{Portable Document Format, a standard format
that ensures consistent display across different computers and media.} for printing
and use on the computer.

\subsection*{The anatomy and organization of the catalog}

The entries in the main catalog listing are ordered by title, with title
cross-references interfiled. Cross-references are not included in the numbering
of entries.

The subjects appearing in the subject index are drawn from the Library of
Congress Subject Headings (LCSH), a cross-referenced list of standardized terms
for describing the subjects of books used by most libraries in the
English-speaking world. Due to the limited time available for the project,
we -- by and large -- did not assign subject headings in the course of our
cataloging, relying instead on the subjects assigned by other libraries. As
such, the subject index reflects the subjects assigned by non-special libraries,
so overly broad subjects such as ``Birds'' may be overrepresented.

Although conceptually straightforward, the geographical index may require a few
words. The geographical locales listed in the index are drawn from the subject
headings, as assigned by other libraries. Particularly for older books, this
means that some out-of-date or idiosyncratic terminology may be used. Due to the
North America-centric nature of the Library of Congress Subject Headings, U.S.
states and Canadian provinces are posted independently (for example, a book about
Virginian birds would appear in this index under ``Virginia'' rather than ``United
States--Virginia'').

Of all the indices, the series index is both potentially the most useful and
the most confusing. Due to a lack of standardization in the library community
on how to represent series in a bibliographic record, there are a number of
inconsistencies that the reader must be aware of. The most obvious is the
frequent duplication of entries differing only in the use of an abbreviation
(for example, one entry may abbreviate the word ``volume'' as ``vol.'', and the
next, ``v.''). Slightly less obvious are the inconsistencies in grouping caused
by differing standards of entry. In some records, for example, the publication
numbers may be included in the main entry, obscuring the relationship between that
work and others in the same series. Finally, in the interest of readability the
label ``(no.)'' was added before all numbers without labels. Thus, the citation
for such an item should be (for example) ``BTO guide 17'' rather than ``BTO guide
(no.) 17.''
