\chapter*{Introduction}
\addcontentsline{toc}{chapter}{Introduction} 

The cataloging of this collection was completed by Jared Camins-Esakov and Catey
Farley with the help of Sarah Simms over a three month period between January
and March 2009. The cataloging was done in
{\ddag}biblios.net\footnote{http://www.biblios.net/}, a free service provided by
LibLime\footnote{LibLime, a provider of open source software for libraries,
http://www.liblime.com/} that enabled us to use catalog records from other
libraries as a starting point rather than needing to enter in all the common
data (e.g., title, author, series) ourselves.

\subsection*{The anatomy and organization of the catalog}

The subjects appearing in the subject index are drawn from the Library of
Congress Subject Headings (LCSH), a cross-referenced list of standardized terms
for describing the subjects of books used by most libraries in the
English-speaking world. Due to the limited time available for the project,
we -- by and large -- did not assign subject headings in the course of our
cataloging, relying instead on the subjects assigned by other libraries. As
such, the subject index reflects the subjects assigned by non-special libraries,
so overly broad subjects such as ``Birds'' may be overrepresented.

Although conceptually straightforward, the geographical index may require a few
words. The geographical locales listed in the index are drawn from the subject
headings, as assigned by other libraries. Particularly for older books, this
means that some out-of-date or idiosyncratic terminology may be used. Due to the
North America-centric nature of the Library of Congress Subject Headings, U.S.
states and Canadian provinces are posted independently (for example, a book about
Virginian birds would appear in this index under ``Virginia'' rather than ``United
States--Virginia'').

Of all the indices, the series index is both potentially the most useful and
the most confusing. Due to a lack of standardization in the library community
on how to represent series in a bibliographic record, there are a number of
inconsistencies that the reader must be aware of. The most obvious is the
frequent duplication of entries differing only in the use of an abbreviation
(for example, one entry may abbreviate the word ``volume'' as ``vol.'', and the
next, ``v.''). Slightly less obvious are the inconsistencies in grouping caused
by differing standards of entry. In some records, for example, the publication
numbers may be included in the main entry, obscuring the relationship between that
work and others in the same series. Finally, in the interest of readability the
label ``(no.)'' was added before all numbers without labels. Thus, the citation
for such an item should be (for example) ``BTO guide 17'' rather than ``BTO guide
(no.) 17.''
