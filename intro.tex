\chapter*{Introduction}
\addcontentsline{toc}{chapter}{Introduction} 

\subsection*{The Weinberger ornithology collection}
\pagestyle{plain}

As of January 2009, Marc Weinberger, an avid book collector and bird watcher, had
built a book collection of more than eight thousand volumes. Roughly one quarter
of the collection focused on ornithology and bird watching. This catalog describes
the core of that ornithology collection, some 1900 volumes kept in the library at
the Weinberger residence in New Rochelle, N.Y. until April 2009 when it was
relocated to his new residence in Petaluma, California.

The especial strength of the collection is in field guides, with secondary emphasis
on reference works and ornithological catalogs of the twentieth century.
The collection covers all parts of the globe, from Antarctica to the Antilles and
India to Indiana, with particular emphasis on American birds.

Of the 1317 works (many of which are multi-volume sets) listed in this catalog,
there is not one which does not deserve at least some brief mention, but -- in
the interests of space -- we must content ourselves with listing just a few of
the highlights:
\begin{itemize}
\item No. 94 of 300 signed copies of Walter Rothschild's \textit{Extinct birds},
	the very copy owned by both John Eliot Thayer and Rodolphe Meyer de Schauensee (\ref{670})
\item A first edition of Roger Tory Peterson's influential \textit{Field Guide to
	the Birds} (\ref{723})
\item No. 148 of 350 copies printed of the \textit{Sunset Edition De Luxe} (one
	of the less common editions of this beautiful work) of William Leon Dawson's
	\textit{Birds of California} (\ref{220})
\item Several items from the collection of the important ornithologist Rodolphe
	Meyer de Schauensee (\ref{543}-\ref{544}, \ref{638}, \ref{670}, \ref{893}, \ref{943})
\item Limited edition of Herbert L. Stoddard's \textit{Bobwhite Quail}, including
	a pencil-signed etching by Frank W. Benson (\ref{511})
\item Several items from the collection of the noted ornithologist Emmet Reid
	Blake (\ref{231}, \ref{327}, \ref{553}, \ref{588}, \ref{634}, \ref{907},
	\ref{909}, \ref{1023}, \ref{1094}, \ref{1158})
\item The earliest work in the collection, the 1831 edition of Alexander Wilson's
	\textit{American Ornithology} (\ref{23})
\item David Armitage Bannerman's esteemed \textit{Birds of the British Isles} (\ref{398})
\end{itemize}

\subsection*{The organization of the catalog}

The entries in the catalog are in title order, with cross-references where a
book might be looked up under more than one title.

The subjects index uses Library of Congress Subject Headings (LCSH), a
cross-referenced list of standardized terms for describing the subjects of books
used by most libraries in the English-speaking world. Because the subject headings
are intended for general libraries -- and wherever possible we used the subject
headings that other libraries assigned -- there are a number of inappropriately
broad subject headings like ``Birds.''

The geographical index likewise uses Library of Congress Subject Headings only
the geographical component). Particularly for some of the older books in the
collection, this means that some out-of-date or idiosyncratic terminology may be
used. For example, ``Zaire'' instead of the ``Democratic Republic of the Congo''
or ``Burma'' instead of ``Myanmar.'' Also, because the Library of Congress
Subject Headings are North America-centric, books about U.S. states and Canadian
provinces are listed alphabetically in the index rather than grouped under
``United States'' and ``Canada'' respectively. For example, ``Alabama'' instead
of ``United States--Alabama.''

Of all the indexes, the series index is both potentially the most useful and
the most confusing. Due to a lack of standardization in the library community
on how to represent series in a bibliographic record, there are several
inconsistencies. Often there will be two (or more) entries differing only in
the use of an abbrevation (for example, one entry may abbreviate the word
``volume'' as ``vol.'', and the next, ``v.''). Slightly less obvious are
inconsistencies in grouping. In some entries, the volume/issue number may not
be treated as a subentry of the series name, potentially causing some
volumes/issues to appear in the index out-of-order.  Finally, in order to make
the index more readable, the label ``(no.)'' was added before all volume/issue
numbers without labels. Thus, the citation for such an item should be (for
example) ``BTO guide 17'' rather than ``BTO guide (no.) 17.'' Any volume or issue
labels not in parentheses should be included in the citation (for example
``Antarctic research series, v. 12'').

\subsection*{Catalogers' notes}

The cataloging of this collection was completed by Jared Camins-Esakov and Catey
Farley over a three month period between January and March 2009. The cataloging
was done using {\ddag}biblios.net\footnote{\url{http://www.biblios.net/}}, a free
service from LibLime\footnote{LibLime, a provider of open source software
for libraries, \url{http://www.liblime.com/}} that enabled us to use catalog
records from other libraries as a starting point rather than needing to enter in
all the common data (e.g., title, author, series) ourselves.

Following the completion of the cataloging, we converted the records into MARCXML,
a standard administered by the Library of
Congress\footnote{\url{http://www.loc.gov/standards/marcxml/}}. The records were
then converted from MARCXML to \LaTeX\footnote{An open source system supported by
the \LaTeX project, \url{http://www.latex-project.org/}}, a document preparation
system for high-quality typesetting system, using XSLT\footnote{eXtensible
Stylesheet Language Transformations, \url{http://www.w3.org/TR/xslt20/}}, a
technology for transforming data in XML\footnote{See \emph{A Gentle Introduction
to XML} from the Text Encoding Initiative at
\url{http://www.tei-c.org/Guidelines/P4/html/SG.html} for more information about
XML.} formats such as MARCXML to other formats (in this case, \LaTeX). The document
was then typeset into a PDF\footnote{Portable Document Format, a standard format
that ensures consistent display across different computers and media.} for printing
and use on the computer.

