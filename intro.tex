\chapter*{Introduction}
\addcontentsline{toc}{chapter}{Introduction} 

\subsection*{The organization of the catalog}

The entries in the catalog are in title order, with cross-references where a
book might be looked up under more than one title.

The subjects index uses Library of Congress Subject Headings (LCSH), a
cross-referenced list of standardized terms for describing the subjects of books
used by most libraries in the English-speaking world. Because the subject headings
are intended for general libraries -- and wherever possible we used the subject
headings that other libraries assigned -- there are a number of inappropriately
broad subject headings like ``Birds.''

The geographical index likewise uses Library of Congress Subject Headings only
the geographical component). Particularly for some of the older books in the
collection, this means that some out-of-date or idiosyncratic terminology may be
used. For example, ``Zaire'' instead of the ``Democratic Republic of the Congo''
or ``Burma'' instead of ``Myanmar.'' Also, because the Library of Congress
Subject Headings are North America-centric, books about U.S. states and Canadian
provinces are listed alphabetically in the index rather than grouped under
``United States'' and ``Canada'' respectively. For example, ``Alabama'' instead
of ``United States--Alabama.''

Of all the indexes, the series index is both potentially the most useful and
the most confusing. Due to a lack of standardization in the library community
on how to represent series in a bibliographic record, there are several
inconsistencies. Often there will be two (or more) entries differing only in
the use of an abbrevation (for example, one entry may abbreviate the word
``volume'' as ``vol.'', and the next, ``v.''). Slightly less obvious are
inconsistencies in grouping. In some entries, the volume/issue number may not
be treated as a subentry of the series name, potentially causing some
volumes/issues to appear in the index out-of-order.  Finally, in order to make
the index more readable, the label ``(no.)'' was added before all volume/issue
numbers without labels. Thus, the citation for such an item should be (for
example) ``BTO guide 17'' rather than ``BTO guide (no.) 17.'' Any volume or issue
labels not in parentheses should be included in the citation (for example
``Antarctic research series, v. 12'').
